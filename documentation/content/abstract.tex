%!TEX root = ../doc.tex
\chapter*{Abstract}
\label{sec:Abstract}
Industrieroboter werden von diversen Herstellern angeboten, diese Roboter besitzen meist eine proprietäre Steuerung und Bahnplanung. Jeder Hersteller von Robotern und den entsprechenden Steuerungen bietet in der Regel ein sehr ähnliches Funktionspaket an. Der Aufruf dieser Funktionen und die dazu benötigte Programmiersprachen sind nicht bei allen Herstellern gleich. Bis anhin sind durch die Hersteller der Roboter keine Bestrebungen ersichtlich eine vereinheitlichte Programmiersprache und Ansteuerung der Roboter zu entwickeln.
Mit dem Robot Operating System (ROS) und der dazugehörigen Erweiterung für Industrielle Roboter (ROS-Industrial) ist ein Framework vorhanden, welches es ermöglicht universell Software und Applikationen in der Robotik zu entwickeln. \\

Ziel dieser Arbeit ist es die Möglichkeiten und Einschränkungen von ROS-Industrial zu ermitteln, dazu soll eine Entwicklungsumgebung mit ROS-Industrial aufgesetzt werden und anschliessend eine Montageaufgabe mit ROS-Inudstrial realisiert werden. Als Montageaufgabe ist das Zusammenbauen eines Kugelschreibers, im Industie 4.0 Demonstrator vorgesehen. Der Industrie 4.0 Demonstrator ist ein Projekt der ZHAW und diverser Industriepartner. Es soll die vielen Aspekte von Industrie 4.0 in einem einzelnen Demonstrator veranschaulichen. Die Montageaufgabe soll auf den drei Robotern ABB IRB120, Stäubli TX2-60l und Universal UR3 realisiert werden.\\

Resultat dieser Arbeit ist eine fertig Aufgesetzte Arbeitsumgebung für ROS und ROS-Industrial, und eine Implementierung der Montage Aufgabe in den Industrie 4.0 Demonstrator. Es konnten jedoch nicht alle in der Aufgabenstellung geforderten Ziele erreicht werden. Hauptgründe dafür sind ein noch nicht angelieferter Roboter von Stäubli, Komplikationen bei der Ansteuerung der SMC Aktuatoren über EtherCAT und falsch bestellte Teile. Diese Komplikationen verhinderten, dass die implementierte Montageaufgabe vollständig geprüft und verbessert werden konnte. \\

Die Möglichkeiten von ROS-Industrial sind extrem Umfangreich, es werden durch die Open Source Community eine Vielzahl an Paketen mit Funktionen angeboten. Diese ermöglichen es in sehr kurzer Zeit eine sehr Komplexe Aufgabe zu lösen. 
Die Open Source Community ist zugleich aber auch ein Grund für diverse Einschränkungen von ROS und ROS-Industrial. Da viele Pakete nicht oder nur schlecht gewartet werden kommt es oft vor, dass nicht vollständig funktionsfähige Pakete implementiert werden müssen und diese allenfalls gepatcht werden müssen. Es kann zudem auch vorkommen, dass Nodes aufgrund von nicht nachvollziehbaren Gründen neu gestartet werden müssen.
