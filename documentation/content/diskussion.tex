%!TEX root = ../doc.tex
\chapter{Diskussion und Ausblick}
\label{sec:DiskussionUndAusblick}

\section{Erreichung der Ziele}
Nicht alle gesteckten Ziele für diese Arbeit konnten erreicht werden. Es konnte eine Arbeitsumgebung für ROS und ROS-Industrial eingerichtet werden, mit welcher die Implementierung der Montageaufgabe in den Industrie 4.0 Demonstrator durchgeführt werden konnte. Zur Implementierung der Montageaufgabe musste ein sehr breites Basiswissen von ROS und ROS-Industrial angelegt werden. Dies ermöglicht nun fundierte Aussagen über die Möglichkeiten und Einschränkungen von ROS-Industrial machen zu können.\\

Aufgrund von falsch gezeichneten und falsch bestellten Teilen durch die beiden Bachelorstudenten, welche das mechanische Konzept der Anlage entwickelten, konnte erst kurz vor Abgabe erste Tests mit dem Greifer am Roboter gemacht werden. Dies nach hinten verschoben Testphase verunmöglichte es grosse Änderungen an der Implementation zu machen. Da zum Zeitpunkt der Abgabe der Arbeit die Aktuatoren von SMC noch nicht über EtherCAT angesteuert werden konnten, konnte noch kein Kugelschreiber montiert werden.

\section{Ausblick}
Sobald der Stäubli Roboter geliefert wird und dieser in der Anlage verbaut ist, kann die Implementierung von ROS-Industrial auf dessen Steuerung beginnen. Die Weiterführung von ROS-Industrial als Steuerung des Knickarmroboters ist meines erachtens nur sinvoll, falls in weiterer Zukunft an der ZHAW oder am IMS im Sinne der Forschung weiter mit ROS gearbeitet werden will. Oder falls die Komplexität der Montageaufgabe sich erhöht, weil zum Beispiel über Vision Sensoren Objekte im Arbeitsraum erkennt werden können und dynamisch umfahren werden sollen. \\
Ansonsten ist die Implementierung des momentanen Montageprozesses schneller und effizienter mit den bisherigen Robotertools.
