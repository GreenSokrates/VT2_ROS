%!TEX root = doc.tex
\documentclass[
german,
	10pt, 							% The default document font size, options: 10pt, 11pt, 12pt
	%oneside, 						% Auskommentieren für einseitige Dokumentstruktur
	singlespacing,					% Zeilenabstand - Alternativen: onehalfspacing or doublespacing
	%nolistspacing, 				% Falls onehalfspacing oder doublespacing Auskomentieren!
	%parskip, 						% Auskommentieren für Einzug nach Absatz
	headsepline, 					% Linie unter Kopfzeile
]{scrbook} 							% The class file specifying the document structure


% ====================================================================
% ==						 Packages								==
% ====================================================================
%%% Dokumentinformationen %%%
\usepackage[
	pdftitle={\myTitle},			% Pdf-Titel
	pdfsubject={},
	pdfauthor={\myAuthorName},		% Author
	pdfkeywords={},	
	hidelinks						% Links nicht einrahmen
]{hyperref}

%%% Standard Packages %%%
\usepackage[utf8]{inputenc}
\usepackage[T1]{fontenc}
\usepackage{graphicx, subfigure}	% Einbinden von Grafiken mit \includegraphics
\usepackage{wrapfig}				% Einbinden von Grafiken mit 
\usepackage{lmodern}
\usepackage{color}					% Für Farben z.B. in Tabellen
\usepackage{transparent}			% Für transparente Farben
\usepackage{colortbl}				% für Farben in Tabellen z.B. mit \rowcolor{Gray}
\usepackage{blindtext}				% für Beispieltexte
\usepackage{float}					% Für [H] --> Hier Befehl für Grafiken und Tabellen usw.
\usepackage{geometry}				% Benötigt für Ränder	
%\usepackage[nswissgerman]{babel}
\usepackage[ngerman]{babel}

\usepackage[
	acronym,
	footnote,
	toc,
	section,
	nopostdot,
	nonumberlist
]{glossaries}

%%% Math Packages %%%
\usepackage{amsfonts}				% zusätzliche Schriftzeichen der American Mathematical Society
\usepackage{amsmath}				% mathematische darstellung gem AMS
\usepackage{amssymb} 				% mathematische Symbole aus AMS
\DeclareMathSymbol{*}{\mathbin}{symbols}{"01} % Punkt statt stern bei Multiplikationen

%%% Usefull Packages %%%
\usepackage{url}					% Einbinden von URL's in Text
\usepackage[squaren]{SIunits}		% Zur benutzung von SI-Einheiten
\usepackage{longtable}				% Ermöglicht Tabellen über mehrere Seiten
\usepackage[table,dvipsnames]{xcolor}
							% Für Glossar und Nomenkaltur (evtl ersetzen mit Glossary-Package)
\usepackage{enumitem}				% Zur auflistung mit Zahlen statt punkten

%%% Styling Packages %%%
\usepackage{colortbl}				% Farben in Tabellen
\usepackage{tabularx}				% Bessere Darstellung von Tabellen mit \tabularx
\usepackage{array}					% Benötigt von tabularx
\usepackage[absolute]{textpos}		% Absolute Text Positionierung
\usepackage{scrlayer-scrpage}		% Für Headed/Footer
\usepackage{textcomp}				% Für div Zeichen z.B. Copyright
%\usepackage[scaled]{arial}			% Arial Font

%%% Plots, Mindmaps etc %%%
%\usepackage{tikz}
%\usetikzlibrary{mindmap}


%%% Import Coding %%%
\definecolor{BGgray}{gray}{0.95}	%Hintergrundfarbe für code
\usepackage[newfloat]{minted}
\setminted{
	breaklines=true,				% Automatic Line Breaking
	frame=lines,					% Lines on top and Bottom of Frame
	framesep=1mm,					% Frame Separation 
	baselinestretch=1.1,			% Interlining of the code
	bgcolor=BGgray,					% Sets background color
	fontsize=\footnotesize,			% Sets Fontsize
	%linenos=true,					% Enables Line Numbers
	%showspaces=true,				% Enables sign for spaces
	obeytabs=false,
	tabsize=2,
}
\usemintedstyle{default}
%%% Definition des Verzeichnisses für den Code
\usepackage{listings}
\renewcommand{\lstlistingname}{Sourcecode}
\renewcommand{\lstlistlistingname}{\lstlistingname verzeichniss}

%%% ZHAW Titelpage %%%
\usepackage{wallpaper} 				% Bild hinter Text laden

%%% Index und Referenzen %%%
\usepackage[super,square]{natbib}	% für BibTeX Literaturverzeichnis
\usepackage{makeidx}				% Benötigt für Index

%%% PDF-Import %%%
\usepackage{pdflscape}				% Drehen von PDF's
\usepackage{pdfpages}				% Einzelne PDF-Seiten importieren

% ====================================================================
% ==						Document Styling						==
% ====================================================================

%%% Definition des Textbereichs %%%
\geometry{
	a4paper,
	top=30mm,
	left=30mm,
	right=30mm,
	bottom=30mm,
	headsep=10mm,
	footskip=10mm
}

%%% Definition der Textposition %%%
\setlength{\TPHorizModule}{30mm}
\setlength{\TPVertModule}{\TPHorizModule}
\textblockorigin{10mm}{10mm} 		% start everything near the top-left corner
\setlength{\parindent}{0pt}

%%%% Horizontale Linien auf Titelseite %%%
\newcommand{\HRule}{\rule{\linewidth}{0.5mm}}

%reference to source items inlc source number
\newcommand{\srcref}[1]{\nameref{src:#1} \cite{#1}}

%%% Kopf- und Fußzeile %%%
%\lohead{\myAuthorNames}
\rohead{\myTitle}
\pagestyle{scrheadings}

%%% Quellverzeichnis Style %%%
%\bibliographystyle{abbrv}
\bibliographystyle{ieeetr}

%%% Definition von Farben %%%
\definecolor{gray}{rgb}{0.95,0.95,0.95}
\definecolor{darkgray}{rgb}{0.4,0.4,0.4}

% ====================================================================
% ==				Sonstige Einstellungen							==
% ====================================================================
%%% Besondere Trennungen %%%
\hyphenation{De-zi-mal-tren-nung}

%%% Pfad für Bilder %%%
\graphicspath{{images/}}				% Gibt Pfad für Bilder an
%%% Neue Fig bennenung %%%
\numberwithin{figure}{chapter}

%%% Abb. statt Abbildung %%%
\addto\captionsngerman{
	\renewcommand{\figurename}{Abb.}
	\renewcommand{\tablename}{Tab.}
}